\documentclass[12pt]{article}
\usepackage{setspace}
\usepackage{amsmath}
\usepackage{amssymb}
\usepackage{amsthm} 
\usepackage[dvips,letterpaper,margin=1in,bottom=1in]{geometry}
\usepackage[T1]{fontenc}
\usepackage{setspace}
\usepackage{newtx}
\usepackage{upgreek}
\usepackage{fancyhdr}
\usepackage{lipsum}% just to generate some text
\newcommand{\C}{\mathbb{C}}
\newcommand{\R}{\mathbb{R}}
\newcommand{\N}{\mathbb{N}}
\newcommand{\Q}{\mathbb{Q}}
\newcommand{\Z}{\mathbb{Z}}
\newcommand{\K}{\mathbb{K}}
\newcommand{\g}{\mathfrak{g}}
\newcommand{\gl}{\mathfrak{gl}}
\newcommand{\sun}{SU (n, \C)}
\newcommand{\son}{SO (n, \R)}
\newcommand{\Om}{\Omega}
\newcommand{\V}{\Vert}
\newcommand{\ihat}{\boldsymbol{\hat{\textbf{\i}}}}
\newcommand{\jhat}{\boldsymbol{\hat{\textbf{\j}}}}
\newcommand{\khat}{\boldsymbol{\hat{\textbf{k}}}}
\usepackage{hyperref}
\theoremstyle{definition}
\newtheorem{them}{Theorem}
\newtheorem{prop}[them]{Proposition}
\theoremstyle{definition}
\newtheorem{lem}[them]{Lemma}
\theoremstyle{definition}
\newtheorem{rmk}[them]{Remark}
\theoremstyle{definition}
\newtheorem{defn}[them]{Definition}
\theoremstyle{definition}
\newtheorem{ex}[them]{Example}
\theoremstyle{definition}
\newtheorem{cex}[them]{Counter-example}
\theoremstyle{definition}
\setlength{\parindent}{0cm}

\fancyhf{}
\fancyhead[L]{Samuel Harshe}
\fancyhead[R]{MATH 480}
\fancyhead[C]{\rule[-3ex]{0pt}{3ex} \textbf{Lie Groups \& Lie Algebras}}
\renewcommand\headrulewidth{0.5pt}
\pagestyle{fancy}

\begin{document}
\par{Because I covered much less ground than I had hoped to in my lecture on Monday, I have stated several crucial results about the matrix exponential function in this handout. Taking these results as given, my presentation will get straight to its most important results, establishing the correspondence between Lie groups and Lie algebras.}

\begin{rmk} All matrices in this lecture are
complex unless otherwise noted. Nearly all results
hold for real matrices as well, though.
\end{rmk}

\par{We begin with a statement of Ado’s Theorem, which I stated correctly last class only after a helpful correction from Alex. This is an important result in demonstrating the power of matrix Lie theory, but it will not be employed anywhere in this presentation.}
\begin{them} Every finite-dimensional Lie algebra $\g$
over a field $\K$ of characteristic zero is
isomorphic to a Lie algebra of square matrices under
the commutator bracket.
\end{them}

\begin{them} Several important properties of
the matrix exponential function.
\begin{enumerate}
    \item $\exp(0)= I$.
    \item If $XY = YX$, $\exp(X + Y) = \exp(X)\exp(Y) = \exp(Y)\exp(X)$.
    \item ${\exp(X)}^{-1} = {\exp(-X)}$.
    \item $\exp((\alpha +\beta)X) = \exp(\alpha X)
    \cdot \exp(\beta X)$ for $\alpha, \beta \in
    \C$. 
    \item $\forall C \in GL(n)$, $\exp(CXC^{-1}) = C\exp(X)C^{-1}$.
\end{enumerate}
\end{them}
\par{Note that (3) implies that $\exp(X) \in GL(n)$.}

\begin{them}
Let $X \in M(n)$. Then for $t \in \R$,
$\exp(tX)$ is a smooth curve in
$M(n)$, and 
\[
    \frac{d}{dt} \exp(tX) = X\exp(tX) = \exp(tX)X.
\] This implies
\[
    \frac{d}{dt} \exp(tX) \Big|_{t=0} = X,
\] since $\exp(0) = I$.
\end{them}

\begin{them}
    For all $X, Y \in M(n)$, we have
\[
    \exp(X+Y) = \lim_{m\to\infty} (\exp(X/m)\exp(Y/m))^m.
\]
\end{them}

\begin{them} $\forall X \in M(n)$,
\[
    \det(\exp(X)) = e^{tr(X)}.
\]
\end{them}

\vspace{6pt}
\par{With the matrix exponential and logarithm functions defined and partially described, we can now define one-parameter subgroups, which are used in generating the Lie algebra of a Lie group.}
\begin{defn} A one-parameter subgroup
of $GL(n)$ is a group homomorphism $\gamma: \R^{+} \to
GL(n)$. This implies the following: 
\begin{itemize}
    \item $\gamma$ is continuous.
    \item $\gamma(0) = I$.
    \item $\gamma(t+s) = \gamma(t)\gamma(s)$.
\end{itemize}
\end{defn}

\begin{them} If $\gamma$ is a one-parameter
subgroup of $GL(n)$, there exists a unique $n
\times n$ matrix $X$ such that $\gamma(t) = \exp(tX)$.
\end{them}

\begin{prop} The exponential map $\exp(X)$ is smooth
(infinitely differentiable).
\end{prop}
\par{We have already
proven that if we fix $X$, then $\exp(tX)$ is a
smooth function on $\R$. The present proposition
is different: we are proving that we can take the directional
derivative in the direction of an arbitrary
matrix, which is stronger than taking the
derivative with respect to the parameter $t$.}

\begin{prop} For $X \in GL(n)$,
$\exists A \in M(n): \exp(A) = X$.
\end{prop}
\par{This is interesting but
never used in the lecture.}

\end{document}