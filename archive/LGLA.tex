\documentclass[12pt]{article}
\usepackage{setspace}
\usepackage{amsmath} % AMS Math Package
\usepackage{amssymb} % Math symbols such as
\usepackage[dvips,letterpaper,margin=1in,bottom=1in]{geometry}
\usepackage[T1]{fontenc}
\usepackage{mathptmx}
\usepackage{tocloft}
\usepackage{titlesec}
\usepackage{setspace}
\newcommand{\C}{\mathbb{C}}
\newcommand{\R}{\mathbb{R}}
\newcommand{\N}{\mathbb{N}}
\newcommand{\Q}{\mathbb{Q}}
\newcommand{\Z}{\mathbb{Z}}
\newcommand{\K}{\mathbb{K}}
\newcommand{\flift}{f\uparrow^G}
\newcommand{\glift}{g\uparrow^G}
\newcommand{\fproj}{f\downarrow_{G/ H}}
\newenvironment{remark}{\textbf{Remark.}}{\vspace{6pt}}
\newenvironment{example}{\textit{Example \theexample.}}{}
\newcounter{example}
\newenvironment{definition}{\textbf{Definition.}}{\vspace{6pt}}
\newenvironment{theorem}{\textbf{Theorem.}}{\vspace{6pt}}
\newenvironment{lemma}{\textbf{Lemma.}}{\vspace{6pt}}

\setlength{\parindent}{0pt}
\setlength{\parskip}{6pt}
\renewcommand\cftsecfont{\selectfont\mdseries}
\renewcommand\cftsecpagefont{\mdseries}
\renewcommand\cfttoctitlefont{\fontfamily{phv}\selectfont\textbf}
\titleformat{\section}{\fontfamily{phv}\selectfont\bfseries}{\thesection}{1em}{}
\titleformat{\subsection}{\fontfamily{phv}\selectfont\bfseries}{\thesection}{1em}{}

\title{Lie Groups and Lie Algebras.}
\author{Samuel Harshe}

\begin{document}

\makeatletter
\begin{titlepage}
    \begin{center}
        \vspace*{2.5in}
        {\fontfamily{phv}\selectfont \huge
        \bfseries \@title}\\
        \vspace*{2.5in}
        \onehalfspacing%
        \@author\\
        Yale University\\
        Professor Igor Frenkel\\
        MATH 480: \textit{Senior Seminar} \\
        \today \\
    \end{center}
\end{titlepage}

\onehalfspacing%
\tableofcontents
\vspace{6pt}
\noindent\rule{\textwidth}{1.5pt}

\section{Introduction.}
\par{Before we go nose to grindstone and deal
carefully with the details of Lie groups and Lie
algebras, it merits sketching the telos of this
lecture, both to motivate and to allow readers to
try to anticipate significant connections
throughout the presentation. We therefore begin
with the most important definition: a \textbf{Lie
group} is a set endowed with both group structure
and \textbf{smooth manifold} structure with smooth
inversion and composition maps. This might not
sound much stronger than the definition of a
topological group, yet Lie groups are much nicer
than generic topological groups. We construct a
tangent space to the identity of the Lie group,
give it a bit of extra structure, and call it a
Lie algebra. Then, we can do almost all our work
on the Lie algebra (which conveniently, is
linear), reducing the problem of operating on our
group to a much simpler linear algebra problem.} 

\section{Smooth manifolds.}
\subsection{Definitions.}
\par{A set $X$ equipped with a function
$\mathcal{N}(x)$ that assigns to each $x \in X$ a
nonempty collection of subsets $\{N\}$ (called
\textbf{neighborhoods} of $x$) is a \textbf{topological
space} if it satisfies the following four axioms.}
\begin{enumerate}
    \item Each point $x \in X$  belongs to each of
    its neighborhoods.
    \item Each superset of a neighborhood of $x$
    is also a neighborhood of $x$.
    \item The intersection of any two
    neighborhoods of $x$ is a neighborhood of $x$.
    \item Any neighborhood $N$ contains a
    subneighborhood $M$ such that $N$ is a
    neighborhood of each point in $M$.
\end{enumerate}
\par{A set is \textbf{open} if it contains a
neighborhood around each point.}
\par{The \textbf{basis} of a topological space $M$ is some
family $\mathcal{B}$ of open subsets such that
every open set in $M$ is equal to the union of
some some sub-family of $\mathcal{B}$.}
\par{A topological space is \textbf{second countable}
if it has a countable basis.}
\par{A \textbf{Hausdorff space} is a topological space $M$
where $\forall x, y \in M$, $\exists U_x, U_y$,
neighborhoods of $x$ and $y$, such that $U_x \cap
U_y = \emptyset$.}
\par{A \textbf{neighborhood} of a point $p$ in a topological
space $M$ is any open subset of $M$ containing
$p$.}
\par{A \textbf{open cover} of $M$ is a collection of
open sets in ${U_a}$ whose union $\cup U_a = M$.}
\par{A topological space $M$ is \textbf{locally Euclidean of
dimension $n$} if $\forall p \in M$, $\exists U_p$
neighborhood of $p$ such that there is a
homeomorphism $\phi$ from $U_p$ to an open set in
$\R^n$.}
\par{The pair $(U_p, \phi)$ is called a \textbf{chart}, $U_p$
is called a \textbf{coordinate neighborhood}, and
$\phi$ is called a \textbf{coordinate map}.
A chart $U$ is \textbf{centered at $p$} if
$\phi(p) = 0$.}

\subsection{Topological manifolds.}
We simply concatenate several definitions to
define a \textbf{topological manifold of dimension $n$}, a
Hausdorff, second countable, locally Euclidean
space of dimension $n$.

\begin{example}
    Any open subset of the Euclidean space $S
    \subseteq \R^n$ is a topological manifold with
    chart $(S, id)$.
\end{example}

\begin{example}
    The graph of $y = |x|$ in $\R^2$ is a
    topological manifold of dimension $1$ with the
    coordinate map $(x, |x|) \mapsto x$.
\end{example}

\begin{example}
    The cross $M = \{(x_1,x_2): x_1 = 0$ or $x_2
    = 0\}$ is not a topological manifold.
    Homeomorphisms preserve the number of
    connected components. Observe that $M \setminus \{0\}$
    has $4$ components, whereas $\R^n \setminus \{0\}$ has
    $2$ components if $n=1$ and $1$ component
    otherwise. Therefore there can be no
    homeomorphism from $M$ to an open subset of
    $\R^n$, so $M$ is not a topological manifold.
\end{example}

Two charts $(U, \phi)$ and $(U, \psi)$ are
\textbf{compatible} if $\phi \circ \psi^{-1}$ and
$\psi \circ \phi^{-1}$ are both $C^\infty$.

\par{In other words, we can start in Euclidean
space, go up to the manifold via the inverse of
one map, then come back down via the other map,
all without trouble. Obvserve that if $U \cap V =
\emptyset$, then the functions $\phi \circ
\psi^{-1}$ and $\psi \circ \phi^{-1}$ are
trivially $C^\infty$, so this definition is only
restrictive for charts on nondisjoint
neighborhoods.}

An \textbf{atlas} on a topological manifold $M$ is a
collection $\mathcal{U} = \{(U_a, \phi_a)\}$ of
pairwise compatible charts such that $\cup
\{U_a\}$ forms an open cover of $M$.

An atlas $(U_a, \phi_a)$ is \textbf{maximal} if
it is not the proper subset of any atlas on $M$.

Finally, our main definition: a \textbf{smooth
manifold} is a topological manifold $M$ together
with a maximal atlas.

\begin{example}
    For any smooth function $f: A \to \R^m$ for
    $A$ open in $R^n$, the space $M = \{(x,f(x)):x
    \in A\}$ is smooth $n$-manifold whose atlas
    contains a single chart $(A, \phi)$, with
    coordinate neighborhood $A$, the domain of the
    function, and map $\phi: (x,f(x)) \mapsto x$.
\end{example}

\begin{example}
    The sphere $S^1$ is a $1$-manifold in $\R^2$.
    Define four open neighborhoods $U_+ = \{(x,y)
    \in S^1 : x > 0\}$, $U_{-} = \{(x,y)
    \in S^1 : x < 0\}$, $V_+ = \{(x,y)
    \in S^1 : y > 0\}$, and $V_{-} = \{(x,y)
    \in S^1 : y < 0\}$, with maps
    projecting onto the axis of the variable not
    restricted. Clearly, these neighborhoods
    cover $S^1$, and each is homeomorphic to an
    open subset of $R^1$.
\end{example}

To prove the following theorem, we need a lemma: 

\begin{lemma}
    \textbf{Any open subset of a smooth manifold is a smooth manifold.}
    Let $M^\prime$ be an open subset of $M$ and
    $M$ a smooth manifold with atlas $\{(U_a,
    \phi_a)\}$. Then $\{(U_a \cap M^\prime, \phi_a)\}$ is
    an atlas of $M^\prime$, so $M^\prime$ is
    itself a smooth manifold.
\end{lemma}

\begin{definition}
    $M(n,\K)$ is the set of all $n \times
    n$ matrices with entries in $\K$. (If $\K$ need not
    be distinguished, we write simply $M(n)$, and
    similarly for all other matrix groups.)
\end{definition}

\begin{definition}
    \textbf{$GL(n)$} is the set of all
    $n \times n$ matrices with nonzero determinant. 
\end{definition}

\begin{theorem}
    \textbf{$GL(n, \R)$ is a smooth $n^2$
    manifold.} To begin, we identify $M(n, \R)$
    with $\R^{n^2}$. $\R^{n^2}$ is trivially a
    smooth manifold: take atlas $\{(\R^{n^2},
    id)\}$. Thus, $M(n, \R)$ is a smooth manifold.
    The determinant $\det: GL(n, \R) \to \R
    \setminus \{0\}$ is polynomial in the entries
    of the matrix of which it is taken, so it is
    continuous, and its image is an
    open subset of $\R$. The preimage of any
    continuous map to an open set is open, so the
    preimage of $\det$, namely $GL(n, \R)$, is an
    open subset of $M(n, \R)$. Therefore, by the
    lemma, $GL(n, \R)$ is a smooth manifold of
    dimension $n^2$.
\end{theorem}

\begin{theorem}
    \textbf{$GL(n, \C)$ is a smooth $2n^2$
    manifold.} (This proof is identical to
    \textit{super}, \textit{mutatis mutandis}.) To
    begin, we identify $M(n, \C)$ with
    $\R^{2n^2}$. $\R^{2n^2}$ is trivially a smooth
    manifold: take atlas $\{(\R^{2n^2}, id)\}$.
    Thus, $M(n, \C)$ is a smooth manifold. The
    determinant $\det: GL(n, \C) \to \R^2 \setminus
    \{0\}$ is polynomial in the entries of the
    matrix of which it is taken, so it is
    continuous, and its image is an open subset of
    $\R^2$. The preimage of any continuous map to an
    open set is open, so the preimage of $\det$,
    namely $GL(n, \C)$, is an open subset of $M(n,
    \C$. Therefore, by the lemma, $GL(n, \C)$ is
    a smooth manifold of dimension $2n^2$.
\end{theorem}

\section{Lie groups.}

\subsection{Matrix groups.}
\par{A \textbf{matrix group} is just what it sounds
like: a sets of matrices closed under matrix
multiplication, containing an identity element,
and containing an inverse of each of its elements.
We will begin out treatment of Lie groups by
dealing with matrix groups, which are simple
to work with, then consider how our results
generalize.}

\begin{theorem}
    \textbf{$GL(n)$ is the maximal a matrix group.}
\end{theorem}

\par{\textit{Proof.} Recall from linear algebra
that $\det(AB) = \det(A)\det(B)$. Let $A, B \in
GL(n)$. By the definition of $GL(n)$, $\det(A) \neq
0$ and $\det(B) \neq 0$. Then $\det(AB) =
\det(A)\det(B) \neq 0$, so $AB \in GL(n)$. Also,
$\det(I) \neq 0$ and $AI = IA = A$, $\forall A \in
GL(n)$, so $GL(n)$ contains the identity.
Finally, recall from linear algebra that a square
matrix with nonzero determinant possesses an
inverse. Then $\forall A \in GL(n)$, $\exists
A^{-1}: AA^{-1} = A^{-1}A = I$. This implies
$\det(A^{-1}) = 1/ \det(A)$, so $\det(A^{-1}) \neq
0$, so $A \in GL(n)$.}

\vspace{6pt}
\par{Recall from linear algebra that nonsquare
matrices and matrices with determinants of $0$
lack inverses. Then any set with such a matrix
does not have an inverse for each of its elements,
so it is not a group.}

\par{$\therefore GL(n)$ is a matrix group and any set
of matrices with an element not in $GL(n)$ is not
a matrix group, so $GL(n)$ is the maximal matrix
group. $\square$}

\par{\textbf{}}

\section{Lie algebras.}

\section{Lie groups and Lie algebras together: the exponential and logarithm maps.}

\end{document}